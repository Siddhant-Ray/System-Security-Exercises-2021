 % IMPORTANT: add or remove (comment out) the boolean '\solutiontrue' below to
% create the solution document or the exercise document respectively.
% First we create the switch to make either the exercises or the solutions
\newif\ifsolution\solutionfalse
% To create the solution uncomment '\solutiontrue'
\solutiontrue

\documentclass[a4paper,11pt]{article}
\title{System Security\\
Bot Analysis}

\ifsolution
\author{\bf Solution}
\else
\author{\bf Graded Assignment}
\fi

\ifsolution
\author{Siddhant Ray }
\fi

\usepackage[T1]{fontenc}
\usepackage{ae, aecompl}
\usepackage{a4wide}
\usepackage{boxedminipage}
\usepackage{url}
\usepackage{graphicx}
\usepackage{enumerate}
\usepackage{alltt}
\usepackage{hyperref}

% Some useful commands and environments
\usepackage{framed}
\newenvironment{solution}%
{\par{\noindent\small\textit{Solution:}}\vspace{-12pt}\begin{framed}}%
{\end{framed}\par}



%% tty - for displaying TTY input and output
\newenvironment{tty}%
{\small\begin{alltt}}%
{\end{alltt}}

\begin{document}
\maketitle

In this exercise you will reverse engineer a bot. Bots that are part of a botnet
can be controlled by a botmaster in various ways. In this case the botmaster
uses a chatroom to communicate with one or more bots. Your task is to find the
commands that the botmaster can use and that the bot understands. 

You should run the bot inside the provided VM. This way things will
work as expected and you are protected from possible vulnerabilities of the bot. The VM also contains useful tools for the analysis, such as gdb and Ghidra.

\begin{enumerate}
\item Start the chatroom by running \verb|python3 chatroom.py|. By
default the chatroom will listen for local connections on port 4567. It simply
forwards all messages it receives to all connected parties.
\item Run the chatclient using \verb|python3 chatclient.py|. The client will
be your interface to the chatroom. It allows you to send commands and receive
output.
\item Start the bot, e.g. by running \verb|./bot 127.0.0.1 4567|. This way the
bot will connect to the chatroom. You should see a greeting message from the bot
in your chat client.
\item You can run multiple clients and bots.
\end{enumerate}

\noindent\emph{Note: You are not allowed to answer the questions by referring to the python scripts.}

Reverse Engineering can be done in different ways. Among variations there are
black- and white-box approaches. In black-box approaches the externally
observable data is used. It often provides a good start. Good tools for
black-box analysis are \texttt{strings}, \texttt{strace}, \texttt{lsof},
\texttt{ps}, \texttt{netstat} or \texttt{wireshark}. Especially \texttt{strings}
might be useful for you as it extracts readable strings from the executable.

\subsubsection*{How does the bot communicate with the chat room? Using TCP or
UDP? Is the transmission encrypted? How did you find these answers?
\ifsolution (1 point) \fi}
\ifsolution
\begin{solution}

\begin{verbatim}

The bot communicates using TCP, I checked this using the tcpdump command
and listened for packets on all interfaces of the port 4567. The command 
is given by: 

sudo tcpdump -i any port 4567 -vvv -l -A

Flags : -i any means on all interfaces
        port 4567 specifies the port 
        -vvv is for specifying the ouput as fully verbose
        -l is to list the output in lines
        -A dumps the payload of the packets
        
Output:

14:35:22.636699 IP (tos 0x0, ttl 64, id 9022, offset 0, flags [DF],
proto TCP (6), length 57)
localhost.57486 > localhost.4567: Flags [P.], cksum 0xfe2d (incorrect -> 0xa4ab), seq 1714409768:1714409773, ack 1067720953, win 512, options
[nop,nop,TS val 2469146062 ecr 2468851634], length 5
E..9#>@.@...............f/.(?. ......-.....
.,-..'..hello

I sent the message "hello" to the chat client, and it can clearly be seen
that the message is not encrypted as hello is in plain text in the payload.
\end{verbatim}

\end{solution}\fi

White-box approaches use introspection of the program, e.g. using a debugger.
However, this can be a very time-consuming task. As your task is to identify the
bot commands you do not have to analyse the whole program. Think about where the
received commands will be handled. If you are unfamiliar with network sockets
check which function is used to receive the
data~\footnote{\url{https://en.wikipedia.org/wiki/Berkeley_sockets}}.

You can either set a breakpoint directly after the data is received or attach to
the process when the bot is waiting for new commands and then continue stepping
through the program to observe the data handling.

\subsubsection*{Which system call is used to receive possible commands? Where is the code called (e.g., \texttt{0x08040000})? \ifsolution (1 point) \fi}
\ifsolution
\begin{solution}

\begin{verbatim}

Over here, a combination of listen, select and read is used to receive and
read the commands received.
    
listen (0x32) sets up the socket for listening to and sets up the
file descriptor. This is where the listen function is called: 

0x555555556dae <main+766>       call   0x5555555561b0 <listen@plt>

The select syscall (0x17) is used as a polling resource, it waits till a 
particular file descriptor is available and then, read is called on the
file descriptor once it is ready. This is where the select function is
called: 

0x555555556ea5 <main+1013>      call   0x555555556180 <select@plt>

This can be done in gdb by setting a breakpoint at the line using
break *main + line_number. I guess this line number based on where 
the syscall is executed, which is caught with catch syscall.

Entire flow was :
catch syscall 0x32
catch syscall (verifies we hit 0x17 after listen is invoked)
catch syscall 

This returned main+771 , and then I guessed the line where the code is
called. I know this because I can guess the line which is just a few before
where the code enters the listen() or select() call, and that's where the
function is called.

With that, I determined the memory location where listen and select
were called. disass main was also useful for verifying this information.

\end{verbatim}

\end{solution}\fi


Now you have to understand which commands are accepted by the bot. Check how the
commands are filtered and parsed. Once you think you found a command try it and
observe its functionality.

\subsubsection*{Which commands does the bot accept? List their names, their functionality, describe the output. If external files, such as images are used, include those files into your report. You should understand at least four commands.
\ifsolution (2 points for each command) \fi}

\ifsolution
\begin{solution}
\begin{verbatim}

Using the strings CLI method, most of the commands can be retrieved.

These are the commands the bot accepts: 

    .info - This command gives information about the BOTs ID ,environment 
    of the user (getenv("USER")), hostname (gethostname()), uname(&var),
    release version of the OS and architecture of the machine. 
    
    output of .info -
    bot-3939: syssec, syssec-vm, Linux, 5.11.0-38-generic, x86_64

    .kill - This command is used to shut down all the running bots.
    output of .kill - All the bots are shut down.
    
    .processes - Shows all the running processes currently running along
    with their PIDs.
    output of .processes - 
    bot-3806: 
    PID TTY          TIME CMD
   1958 tty2     00:00:22 Xorg
   2002 tty2     00:00:00 gnome-session-b
   3799 pts/0    00:00:00 python3
   3802 pts/1    00:00:00 python3
   3806 pts/2    00:00:00 bot
   3860 pts/4    00:00:00 bash
   3862 pts/4    00:01:04 java
   3991 pts/4    00:00:00 demangler_gnu_v
   3997 pts/4    00:00:00 decompile
   4280 pts/3    00:00:00 gdb
   4287 pts/3    00:00:00 bot
   4350 pts/2    00:00:00 sh
   4351 pts/2    00:00:00 ps
    
    .flash - flash a png saying "you got hacked!!! Hahaha!". The image is
    written into /tmp and then deleted by the code itself. To retrieve the
    image, I set a breakpoint at the point the image was written but not
    deleted yet.
    output of .flash - an image saying "you got hacked!!! Hahaha!"
   
\end{verbatim}
\end{solution}\fi

\begin{figure}[h]
    \centering
    \includegraphics[width=5cm]{tmp_bot_image.png}
    \caption{Flash image}
    \label{fig:galaxy}
\end{figure}

\subsubsection*{There are certain commands that do not show up in the output of
\texttt{strings}. Why? \ifsolution (1 point) \fi}
\ifsolution

\begin{solution}
\begin{verbatim}
Some commands do not show up as strings as they are stored as character
arrays. An example for this the .secret command, which is stored as a char
array.
\end{verbatim}

\end{solution}\fi


\subsubsection*{Some messages are filtered out first by the bot. Why is this necessary? \ifsolution (1 point) \fi}

\ifsolution
\begin{solution}
\begin{verbatim}
The bot accepts some commands which begin with a "bot-%PID." and shows error messages such as "huh" and "these are not the
commands you are looking for" , if an incorrect command is entered. 
However, if a wrong command is entered which begins with a "bot-%PID.", the user is not shown anything.

These messages are filtered as the bot doesn't know who is sending
the messages, it just listens on a socket. Anything which begins with a "."
, is redirected to the bot and hence, incorrect commands beginning with "."
are filtered else it can create an incorrect infinite loop.
\end{verbatim}
\end{solution}\fi


\subsubsection*{How does a bot generate the name it uses in the chatroom? \ifsolution (1 point) \fi}
\ifsolution
\begin{solution}
\begin{verbatim}
    The name of the bot is from the its PID, concatenated to
    "bot-%i"%PID. After decompiling with ghidra, can see the method: 
    
    local_20 = getpid();
    snprintf(own_name,0x80,"bot-%i",(ulong)local_20);
    
\end{verbatim}
\end{solution}\fi

\subsubsection*{Which other commands can you find? Describe how you found them,
how the bot parses them, and their functionality. You can find up to seven total commands.
\ifsolution (2 points for each command) \fi}

\ifsolution
\begin{solution}
\begin{verbatim}
    Other commands I found: For this I used a combination of checking the
    decompiled files using ghidra and the strings command for the bot
    binary.
    
    .secret - This command if typed many times, changes the background 
    colour to a horrible green, pink or cyan. The bot parses this command
    by reading "secret" as a character array. This doesn't show up in the
    output of strings as it is a char array.
    
    .fight - This commands causes a competition between two running bots.
    Based on the competition, one bot survives and the other one dies.
    The bot parses this command by reading two strings with the bot names.
    More details on the working are in the last answer. 
    
    .e4stere66!1 - This one was tricky. I decompiled the binary using 
    ghidra and saw that the data in "enccmdstr" was XORed with 0x81 using
    func2(), char by char. Since I couldn't find anything else, I guessed
    this must be the command. I took the bytes in enccmdstr and wrote the
    Python script below to decode it.
    
    #!/usr/bin/env python3

    numbers = [ 0xe4, 0xb5, 0xf2, 0xf5, 0xe4, 0xf3, 0xe4, 0xb7, 0xb7, 0xa0,
    0xb0, 0x00 ]
    
    dec_numbers = [int(str(item), 0) for item in numbers]
    print(dec_numbers)
    
    xored_cmd = [chr(item^0x81) for item in dec_numbers]
    print(xored_cmd)
    print(''.join(xored_cmd))
    
    output of the script - e4stere66!1

    output of .e4stere66!1 - Congrats, you found the easteregg. Document
    how you found it and the command for extra points.
     
    
    
\end{verbatim}
\end{solution}\fi

\subsubsection*{The bots can enter into some form of ``conflict''. How do they
communicate? What is therefore required for two bots to communicate? How do they
find a winner? \ifsolution (2 points) \fi}

\ifsolution
\begin{solution}
\begin{verbatim}
    
    There is a command which can cause a "conflict" between the bots, i.e.
    the ".fight bot-PID1 bot-PID2", which causes both the bots to generate
    a random number. Each bot send this random number to the other one over
    a AF_UNIX TCP SOCK_STREAM. This is a protocol which allows interprocess
    communication on the same local machine. Thus, it is required that
    the bots are running on the same local machine.
    
    The conflict is resolved by each bot locally comparing the received
    number against its generated number. The larger number is the winner.
    Each bot then sends its decision to the other bot.
\end{verbatim}
\end{solution}\fi



\subsubsection*{Notes}
\begin{itemize}
\item Normally symbols, such as function names would be stripped out. We left
them in to make this exercise a bit easier.
\item Pasting into the chatclient might not work correctly. Try to add a space
after pasting.
\item When using gdb it can be helpful to use the \verb|layout asm| command.
\item To understand the advanced features, it helps to know about
select\footnote{\url{https://en.wikipedia.org/wiki/Select_\%28Unix\%29}}.
\item Normally only the bot executable would run inside the VM, while the
chatroom and chatclient would run on different machines. However, for this
exercise it is fine to run all inside the VM. If you want to run the chatroom
and chatclient outside the VM, you need to adjust the command line parameters of
bot and possibly chatclient (see \verb|python3 chatclient.py --help|).
\item You can find a maximum of seven commands.
\end{itemize}



\begin{thebibliography}{---}
\bibitem[1]{example1} Linux man page for Select, \url{https://man7.org/linux/man-pages/man2/select.2.html}
\bibitem[2]{example2} ghidra, 
\url{https://ghidra.re/docs/}
\bibitem[3]{example3} Linux man page for tcpdump, 
\url{https://www.tcpdump.org/manpages/tcpdump.1.html}
\bibitem[3]{example4} Some small doubts on StackOverflow,
\url{https://stackoverflow.com/}
\end{thebibliography}

\end{document}

